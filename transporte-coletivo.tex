\section{A crise do transporte coletivo em Florianópolis}

A análise econômica acima nos ajuda a entender a situação do
transporte coletivo de Florianópolis da seguinte no seguinte
raciocínio.

A - Existe concessões exclusivas das linhas de Florianópolis, com a
prefeitura agindo como guardiã das mesmas.

B - Por causa dessa atitude surgem micro monopólios maus no transporte
coletivo na cidade.

C - Os monopólios causam desperdício, mau uso, má qualidade e preços
mais altos do que os encontrados em um mercado livre. (vide sessão
anterior)

Resumindo, o estado garante a posição das empresas de transporte
coletivo (junto com seus sindicatos aparelhados), transferindo os
impostos dos moradores de Florianópolis para elas e mantendo o preço
nas alturas. Um abraço corporativista que garante farto apoio em
épocas de eleições, tanto das empresas quanto dos sindicatos dos
trabalhadores das mesmas, em troca de uma condição confortável e
suculenta para os parasitas sobre rodas.

