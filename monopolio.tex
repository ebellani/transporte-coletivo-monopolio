\section{O que é monopólio}

Se os leitores desejarem uma leitura mais abrangente recomendo a fonte
que usei nas palavras a seguir.\cite{monopoly-bad-good} Ela é muito
mais completa e profissional que a amadora resenha que se segue. Pois
bem. Existem 2 tipos de monopólio, o bom e o mau.

\subsection{O bom}

O que controla o preço de produtos produzidos por monopolistas
\emph{em um ambiente de mercado livre?} Basicamente a concorrência em
potencial, ou seja, mesmo que uma empresa fique tão eficiente que
apenas ela reste num determinado setor, ela não pode relaxar e começar
a cobrar preços ``abusivos'', pois tal atitude criaria espaço para
novos empreendedores e empresas de outros setores adentrarem o setor
do monopolista.

Em outras palavras:

\begin{quote}
  A concorrência potencial existe em todas as áreas da produção e do
  comércio em que haja liberdade de entrada; áreas em que qualquer
  pessoa seja livre para entrar e competir.  Em outras palavras, em
  qualquer setor em que o governo não impeça a livre entrada por meio de
  licenças, concessões, parcerias e outras formas de controle, a
  concorrência potencial irá existir.  As empresas e os empreendedores
  estão continuamente em busca de novos itens e novas linhas de
  produção.  Motivados pela busca do lucro e guiados pelo sistema de
  preços, eles estão constantemente ávidos para empreender em qualquer
  área pouco explorada cujos rendimentos potenciais sejam atipicamente
  altos.

  Não havendo regulamentações e burocracias governamentais, a incursão
  em um outro setor da economia exigirá de uma empresa pouco mais do que
  uma simples reorganização, atualização e aquisição de novos
  equipamentos, algo que pode ser feito em algumas semanas ou meses.
  Ou, no extremo, instalações novas podem ser construídas para se
  empreender uma vigorosa incursão neste novo setor.  Assim, um
  produtor, seja ele um monopolista, um duopolista ou um concorrente
  dentre vários, estará sempre enfrentando a concorrência potencial de
  todos os outros produtores existentes no mercado.
  \cite{monopoly-bad-good}
\end{quote}

\subsection{O mau}

O mau é resultado da negação direta da concorrência em potencial,
através da única entidade capaz de impedir a entrada de novos
empreendedores num determinado mercado, o estado. Se esta entidade
garantir uma existência confortável aos monopolistas com certeza estes
irão se aproveitar de sua posição dominante e enforçar preços muito
mais altos do que os encontrados no mercado livre, além da baixa
qualidade e serviços oferecidos.

Novamente fazendo uso das palavras de \cite{monopoly-bad-good}:


\begin{quote}
  o governo efetivamente restringe a concorrência e cria monopólios
  locais e nacionais.  Toda a regulamentação governamental sobre o
  mercado tem o objetivo de garantir a determinadas empresas — os
  membros do monopólio, oligopólio ou cartel — uma renda ``justa'', o
  que significa uma renda bem maior do que aquela que conseguiriam no
  livre mercado.
\end{quote}

 
